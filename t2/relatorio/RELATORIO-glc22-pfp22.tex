\documentclass[a4paper, 11pt]{article}
\usepackage[top=3cm, bottom=3cm, left = 2cm, right = 2cm]{geometry}
\usepackage[brazilian]{babel}
\usepackage{setspace}
\usepackage{graphicx}

\title{Relatório: Trabalho 2 -- Otimização de Desempenho}
\author{Gabriel Lisboa Conegero -- GRR20221255\\
Pedro Folloni Pesserl -- GRR20220072\\
\textit{Departamento de Informática}\\
\textit{Universidade Federal do Paraná -- UFPR}\\
Curitiba, Brasil\\
\texttt{glc22@inf.ufpr.br, pfp22@inf.ufpr.br}}
\date{}

\begin{document}
\maketitle

\begin{abstract}
\begin{singlespace}
    Este relatório documenta o processo de otimização de um programa que
    realiza ajuste polinomial de curvas, utilizando o método dos
    \textbf{Mínimos Quadrados} e \textbf{Eliminação de Gauss}. Também
    apresenta a comparação entre as duas versões do programa, obtida a
    partir da ferramenta LIKWID.
\end{singlespace}
\end{abstract}

\section{Metodologia da análise}
A análise do programa de ajuste polinomial de curvas foi feita considerando
três seções principais do código, que realizam, respectivamente:
\begin{enumerate}
    \item Geração do sistema linear pelo Método dos Mínimos Quadrados;
    \item Solução do sistema linear pelo Método da Eliminação de Gauss;
    \item Cálculo de resíduos do polinômio encontrado.
\end{enumerate}
Tanto a seção de geração do sistema linear quanto a de cálculo dos resíduos do
polinômio foram avaliadas com as seguintes métricas: tempo de execução, número
de operações aritméticas de ponto flutuante, com e sem uso de SIMD, por segundo
(FLOP/s), banda de memória e taxa de \textit{miss} na cache de dados. A seção de
solução do sistema linear teve seu desempenho avaliado em tempo de execução e
FLOP/s, apenas.

\section{Otimizações realizadas}
\subsection{Geração do sistema linear}
% tem muita coisa pra melhorar nesse parágrafo
A otimização feita foi calcular as potências de cada $x_i$ e ir somando nos
somatórios, dessa forma é otimizado o acesso a cache do vetor de pontos.
Comparando com $v1$ que calculava cada somatório da matriz em sequência, vemos
que a taxa de cache miss é maior em $v1$, pois quando vamos calcular o próximo
somatório temos que recarregar os pontos do vetor.
% \begin{figure}
%     \centering
%     \includegraphics{}
%     \caption{Taxa de cache miss ao gerar sistema linear}
%     \label{fig:gera_SL_L2_miss}
% \end{figure}

\section{Gráficos}

\end{document}

