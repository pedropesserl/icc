\documentclass[a4paper, 11pt]{article}
\usepackage[top=3cm, bottom=3cm, left = 2cm, right = 2cm]{geometry}
\usepackage[brazilian]{babel}
\usepackage{setspace}
\usepackage{graphicx}

\title{Relatório: Trabalho 2 -- Otimização de Desempenho}
\author{Gabriel Lisboa Conegero -- GRR20221255\\
Pedro Folloni Pesserl -- GRR20220072\\
\textit{Departamento de Informática}\\
\textit{Universidade Federal do Paraná -- UFPR}\\
Curitiba, Brasil\\
\texttt{glc22@inf.ufpr.br, pfp22@inf.ufpr.br}}
\date{}

\begin{document}
\maketitle

\begin{abstract}
\begin{singlespace}
    Este relatório documenta o processo de otimização de um programa que
    realiza ajuste polinomial de curvas, utilizando o método dos
    \textbf{Mínimos Quadrados} e \textbf{Eliminação de Gauss}. Também
    apresenta a comparação entre as duas versões do programa, obtida a
    partir da ferramenta LIKWID.
\end{singlespace}
\end{abstract}

\section{Metodologia da análise}
A análise do programa de ajuste polinomial de curvas foi feita considerando
três seções principais do código, que realizam, respectivamente:
\begin{enumerate}
    \item Geração do sistema linear pelo método dos Mínimos Quadrados;
    \item Solução do sistema linear pelo método da Eliminação de Gauss;
    \item Cálculo de resíduos do polinômio encontrado.
\end{enumerate}
Tanto a seção de geração do sistema linear quanto a de cálculo dos resíduos do
polinômio foram avaliadas com as seguintes métricas: tempo de execução, número
de operações aritméticas de ponto flutuante por segundo (FLOP/s), com e sem uso
de SIMD, banda de memória utilizada e taxa de \textit{miss} na \textit{cache}
de dados. A seção de solução do sistema linear teve seu desempenho avaliado em
tempo de execução e FLOP/s, apenas.

\section{Otimizações realizadas}
\subsection{Geração do sistema linear}
Originalmente, a versão 1 do programa utilizava uma tabela de \textit{lookup}
para armazenar as potências, de 0 a $2m$, dos pontos de entrada, onde $m$ é o
grau do polinômio a ser ajustado. Porém, isso precisou ser modificado, porque a
entrada agora pode conter até $10^8$ pontos, o que impossibilita o
armazenamento das potências de todos os pontos na memória. Então, a versão 1
agora utiliza a função \texttt{pow\_inter()} para calcular as potências dos
x's a cada iteração.

A otimização empregada na versão 2 foi inverter a ordem dos laços no cálculo
das potências, de forma que iteramos sobre o vetor de pontos de entrada apenas
uma vez, possibilitando que ele seja mantido (ainda que em parte) em
\textit{cache}. Dessa maneira, percorremos a primeira linha e última coluna da
matriz do sistema linear e o vetor de termos independentes múltiplas vezes,
calculando a próxima potência do ponto atual que estamos somando a cada
iteração com apenas uma multiplicação sobre a potência anterior.

Como o vetor de pontos é muito maior do que a matriz do sistema linear, que é
sempre de ordem 5, espera-se que isso diminua a taxa de \textit{miss} na
\textit{cache} por ser necessário recarregá-lo em \textit{cache} menos vezes.

% \begin{figure}
%     \centering
%     \includegraphics{}
%     \caption{Taxa de \textit{cache} miss ao gerar sistema linear}
%     \label{fig:gera_SL_L2_miss}
% \end{figure}

\section{Gráficos}


\end{document}

