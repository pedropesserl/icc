\documentclass[a4paper, 11pt]{article}
\usepackage[top=3cm, bottom=3cm, left = 2cm, right = 2cm]{geometry}
\usepackage[brazilian]{babel}
\usepackage{setspace}
\usepackage{graphicx}

\title{Relatório: Trabalho 2 -- Otimização de Desempenho}
\author{Gabriel Lisboa Conegero -- GRR20221255\\
Pedro Folloni Pesserl -- GRR20220072\\
\textit{Departamento de Informática}\\
\textit{Universidade Federal do Paraná -- UFPR}\\
Curitiba, Brasil\\
\texttt{glc22@inf.ufpr.br, pfp22@inf.ufpr.br}}
\date{}

\begin{document}
\maketitle

\begin{abstract}
    \begin{singlespace}
        Otimização do código de ajuste de curva polinomial utilizando o método do \textbf{Mínimos quadrados} e \textbf{Eliminação de gauss}. Comparação entre duas versões utilizando a ferramenta Likwid.
    \end{singlespace}
\end{abstract}

\section{Objetivo}
O objetivo deste relatório é relatar o processo de otimização do código e comparar duas versões do mesmo código utilizando as medidas:
\begin{itemize}
    \item Tempo de execução.
    \item Número de operações. aritiméticas em ponto flutuante (Com uso de AVX e sem).
    \item Banda de memória.
    \item Taxa de cache miss de dados.
\end{itemize}
Para os seções do código:
\begin{enumerate}
    \item Geração do sistema linear pelo Método dos Mínimos Quadrados.
    \item Solução do sistema linear pelo Método da Eliminação de Gauss.
    \item Cálculo de resíduo.
\end{enumerate}

\section{Otimizções}
\subsection{Geração do sistema linear}
A otimização feita foi calcular as potências de cada $x_i$ e ir somando nos somatórios, dessa forma é otimizado o acesso a cache do vetor de pontos. Comparando com $v1$ que calculava cada somatório da matriz em sequência, vemos que a taxa de cache miss é maior em $v1$, pois quando vamos calcular o próximo somatório temos que recarregar os pontos do vetor.
\begin{figure}
    \centering
    \includegraphics{}
    \caption{Taxa de cache miss ao gerar sistema linear}
    \label{fig:gera_SL_L2_miss}
\end{figure}


\end{document}

